% Created 2019-02-07 Thu 06:21
% Intended LaTeX compiler: pdflatex
\documentclass[11pt]{article}
\usepackage[utf8]{inputenc}
\usepackage[T1]{fontenc}
\usepackage{graphicx}
\usepackage{grffile}
\usepackage{longtable}
\usepackage{wrapfig}
\usepackage{rotating}
\usepackage[normalem]{ulem}
\usepackage{amsmath}
\usepackage{textcomp}
\usepackage{amssymb}
\usepackage{capt-of}
\usepackage{hyperref}
\usepackage[margin=0.5in]{geometry}
\author{Parker Gabel}
\date{\today}
\title{CS345 Homework 2}
\hypersetup{
 pdfauthor={Parker Gabel},
 pdftitle={CS345 Homework 2},
 pdfkeywords={},
 pdfsubject={},
 pdfcreator={Emacs 26.1 (Org mode 9.1.9)}, 
 pdflang={English}}
\begin{document}

\maketitle
\section{It is in O(n\(^{\text{2}}\)) and \(\Omega\)(n)}
\label{sec:org12f3586}
Let n\(_{\text{0}}\) = 1 and c = 1. 
\subsection{Notice for all n \(\ge\) n\(_{\text{0}}\), if n is odd then f(n) = n \(\le\) n\(^{\text{2}}\) and if n is even then f(n) = n\(^{\text{2}}\) = n\(^{\text{2}}\) so f(n) \(\in\) O(n\(^{\text{2}}\)).}
\label{sec:org226af0c}
\subsection{Notice for all n \textgreater{} n\(_{\text{0}}\), if n is odd then f(n) = n = n and if n is even then f(n) = n\(^{\text{2}}\) \textgreater{} n.}
\label{sec:org3e83882}
\section{Yes. Since big theta is an equvalence relation, it is reflexive. For every function f(n), f(n) \(\in\) \(\Theta\)(f(n))}
\label{sec:orgd9c8c44}
\section{This is reflexive, symmetric, and transitive}
\label{sec:org41c4520}
\subsection{Claim: f(n) \(\in\) \(\Theta\)(f(n))}
\label{sec:org8438db7}
\subsubsection{Subclaim: f(n) \(\in\) O(f(n))}
\label{sec:org74ab111}
Let n\(_{\text{0}}\) be any positive integer and c be 1. Notice that for all n \(\ge\) n\(_{\text{0}}\) , f(n) \(\le\) f(n) (in fact it is strictly equal). So f(n) \(\in\) O(f(n)).
\subsubsection{Subclaim: f(n) \(\in\) \(\Omega\)(f(n))}
\label{sec:orgead8068}
Let n\(_{\text{0}}\) be any positive integer and c be 1. Notice that for all n \textgreater{} n\(_{\text{0}}\) , f(n) \(\ge\) f(n) (in fact it is strictly equal). So f(n) \(\in\) \(\Omega\)(f(n)).
\paragraph Since f(n) is in both O(f(n)) and \(\Omega\)(f(n)), it is \(\Theta\)(f(n)).
\subsection{Claim: If f(n) \(\in\) \(\Theta\)(g(n)) then g(n) \(\in\) \(\Theta\)(f(n))}
\label{sec:org3a01646}
Assume f(n) \(\in\) \(\Theta\)(g(n)). So f(n) \(\in\) O(g(n)) and f(n) \(\in\) \(\Omega\)(g(n)). So there is some n\(_{\text{0}}\) and c such that for all n \(\ge\) n\(_{\text{0}}\) c\(_{\text{1}}\) g(n) \(\le\) f(n) \(\le\) c\(_{\text{2}}\) g(n) for some c\(_{\text{1}}\) and c\(_{\text{2}}\). So f(n) \(\le\) c\(_{\text{2}}\) g(n) = 1/c\(_{\text{2}}\) f(n) \(\le\) g(n). So c\(_{\text{1}}\) g(n) \(\le\) f(n) = g(n) \(\le\) 1/c\(_{\text{1}}\) f(n). So 1/c\(_{\text{2}}\) f(n) \(\le\) g(n) \(\le\) 1/c\(_{\text{1}}\) f(n). So g(n) \(\in\) \(\Theta\)(f(n)) 
\subsection{Claim: If f(n) \(\in\) \(\Theta\)(g(n)) and g(n) \(\in\) \(\Theta\)(h(n)) then f(n) \(\in\) \(\Theta\)(h(n)).}
\label{sec:org0a83033}
Assume f(n) \(\in\) \(\Theta\)(g(n)) and g(n) \(\in\) \(\Theta\)(h(n)). So there exists a,b,c,d \textgreater{} 0 and n \textgreater{} 0 such that ag(n) \(\le\) f(n) \(\le\) bg(n) and ch(n) \(\le\) g(n) \(\le\) dh(n). So f(n) \(\ge\) ag(n) \(\ge\) a(ch(n)) = ac(h(n)) and f(n) \(\le\) bg(n) \(\le\) b(dh(n)) = bd(h(n)). So ac(h(n)) \(\le\) f(n) \(\le\) bd(h(n)). So f(n) \(\in\) \(\Theta\)(h(n)).
\section{2, log\(_{\text{3}}\)(n), log\(_{\text{2}}\)(n), n\(^{\text{2/3}}\), 20n, 4n\(^{\text{2}}\), 3\(^{\text{n}}\), n!}
\label{sec:orge21be12}
\section{100n, 10n, n, 2\(^{\text{100n}}\)}
\label{sec:orgd30e0a6}
\section{}
\label{sec:orgbfef390}
\subsection{lim log(n\(^{\text{2}}\))/(log(n) + 5) = 2. Since this is a constant f(n) = \(\Theta\)(n)}
\label{sec:org1ab3a5e}
\subsection{lim (nlog(n) + n)/log(n) = \(\infty\). So f(n) grows faster so f(n) \(\in\) \(\Omega\)(g(n))}
\label{sec:org722b7bb}
\section{}
\label{sec:org2783b70}
\subsection{\(\Theta\)(n\(^{\text{2}}\)). The loop must run n*n times.}
\label{sec:org0ec48e9}
\subsection{\(\Theta\)(n log(n)). The outer loop runs n times and the inner loop runs log(n) times}
\label{sec:orga9a9038}
\subsection{\(\Theta\)(n log(n)). The outer loop runs log(n) times and the inner loop runs n times}
\label{sec:org3e67350}
\subsection{\(\Theta\)(n\(^{\text{2}}\) log(n)). The outer loop is run n times and the inner loop costs n log(n).}
\label{sec:org72da379}
\subsection{\(\Theta\)(n\(^{\text{2}}\)). For each time the outer loop is run, the inner loop happens a random amount of times but it is quranteed to run i times for each value of i from 1 to n.}
\label{sec:orgb00c463}
\section{}
\label{sec:org11551c6}
\begin{verbatim}
/** @return The position of an element in sorted array A
with value k. If k is not in A, return A.length. */
static int binary(int[] A, int k) {
    if (A[0] > k) {
	return ERROR
    }
    int l = -1;
    int r = A.length; // l and r are beyond array bounds
    while (l+1 != r) { // Stop when l and r meet
	 int i = (l+r)/2; // Check middle of remaining subarray
	 if (k < A[i]) r = i; // In left half
	 if (k == A[i]) return i; // Found it
	 if (k > A[i] && k < A[i + 1]) return i + 1; // k not in array
	 else: l = i // In right half
     }
     return A.length; // Search value not in A
  }
\end{verbatim}
\section{}
\label{sec:org446cae2}
\subsection{n \textgreater{} DE / (P + E), E = 1 and P = 4 and D = 30. So n \textgreater{} 30 / 5 = 6. The break even point is six. When n is less than 6 then the linked list requires less space.}
\label{sec:org3674738}
\subsection{n \textgreater{} DE / (P + E), E = 32 and P = 4, and D = 40. So n \textgreater{} 32*40 / 36 = 1280 / 36 = 320 / 9 = 35.55. So the break even point is 35. When n is less than 35 then the linked list requires less space.}
\label{sec:orgac26b0a}
\section{}
\label{sec:orgb332bd8}
\begin{verbatim}
E x;
for(int i = 0; i < Q.length(); i++) {
    x = Q.dequeue();
    S.push(x);
}
 for(int i = 0; i < S.length(); i++) {
    x = S.pop();
    Q.enqueue(x);
}
\end{verbatim}
\end{document}