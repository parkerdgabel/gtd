% Created 2019-03-28 Thu 09:43
% Intended LaTeX compiler: pdflatex
\documentclass[11pt]{article}
\usepackage[utf8]{inputenc}
\usepackage[T1]{fontenc}
\usepackage{graphicx}
\usepackage{grffile}
\usepackage{longtable}
\usepackage{wrapfig}
\usepackage{rotating}
\usepackage[normalem]{ulem}
\usepackage{amsmath}
\usepackage{textcomp}
\usepackage{amssymb}
\usepackage{capt-of}
\usepackage{hyperref}
\author{Parker Gabel}
\date{\textit{<2019-03-25 Mon>}}
\title{Homework3}
\hypersetup{
 pdfauthor={Parker Gabel},
 pdftitle={Homework3},
 pdfkeywords={},
 pdfsubject={},
 pdfcreator={Emacs 26.1 (Org mode 9.1.9)}, 
 pdflang={English}}
\begin{document}

\maketitle
\section{11.3}
\label{sec:org24d306e}
\subsection{(a)}
\label{sec:orgce7fce7}
\[
\begin{matrix}
 0 & 10 & 0 & 20 & 0 & 2 \\
 10 & 0 & 3 & 5 & 0 & 0 \\
 0 & 3 & 0 & 0 & 15 & 0 \\
 20 & 5 & 0 & 0 & 11 & 10 \\
 0 & 0 & 15 & 11 & 0 & 3 \\
 2 & 0 & 0 & 10 & 3 & 0 \\
\end{matrix}
\]
\subsection{(b)}
\label{sec:orgc41f13f}
Vertex -> (Vertex \#, Weight)\\
\begin{center}
\begin{tabular}{rllll}
1 & (2, 10) & (4, 20) & (6, 2) & \\
2 & (2, 10) & (3, 3) & (4, 5) & \\
3 & (2, 3) & (5, 15) &  & \\
4 & (1, 20) & (2, 5) & (5, 11) & (6, 10)\\
5 & (3, 15) & (4, 11) & (4, 11) & (6, 3)\\
6 & (1, 2) & (4, 10) & (5,3) & \\
\end{tabular}
\end{center}
\subsection{(c)}
\label{sec:orge02da1f}
The adjacency matrix has no overhead for pointers so\\
\(\Theta\)(Matrix) = 2|V\(^{\text{2}}\)| = 2(6\(^{\text{2}}\)) = 2(36) = 72\\
The list does have that overhead so\\
\(\Theta\)(List) = 4|V| + 4|E| = 4(6) + 4|9| = 60\\
So the list uses less space\\
\subsection{(d)}
\label{sec:orgfa9adcd}
\(\Theta\)(Matrix) = 2|V\(^{\text{2}}\)| = 72\\
\(\Theta\)(List) = 4|V| + 2|E|\\
\section{11.4}
\label{sec:orga60b04c}
Tree represented by an adjaceny list.\\
\begin{center}
\begin{tabular}{rlll}
1 & (6, 2) & (2, 10) & \\
6 & (5, 3) & (1, 2) & \\
5 & (5, 15) & (2, 10) & \\
2 & (1, 10) & (3, 3) & (4, 5)\\
4 & (2, 5) & (5, 11) & \\
3 & (2, 3) &  & \\
\end{tabular}
\end{center}
\section{11.14}
\label{sec:orge60a311}
We produce a DFS tree and check if any of the nodes in the tree point to any of\\
its ancestors. We assume an appropriate graph and vertex class has been implemented.\\
\begin{verbatim}
class IsCyclic {
static private boolean _isCyclic(Vertex node, HashSet<String, Boolean> visited,
HashSet<String, Boolean> stack) {
    if (stack.get(node.name())) {
	return true;
    }
    if (visited.get(node.name())) {
	return false;
    }
    visited.get(node.name()) = true;
    stack.get(node.name()) = true;
    for (Vertex child : graph.getChildren(node)) {
	if (_isCyclic(child, visited, stack)){
	    return true;
	}
    }
    stack.get(node.name()) = false;
    return false;
    }

    public static boolean isCyclic(Graph graph) {
	// This function returns a HashSet with the name of each vertex as a key and
	// The bolean is set to false
	HashSet<String, Boolean> visited = newGraphSet(graph);
	HashSet<String, Boolean> stack = newGraphSet(graph);
	for(Vertex vertex : graph.getVertices()) {
	    if (_isCyclic(vertex, visited, stack)) {
		return true;
	    }
	}
	return false;
    }
}
\end{verbatim}
Notice this has the same complexity as a DFS so it is \(\Theta\)(|V| + |E|).\\
\section{11.17}
\label{sec:orgc136b78}
3 -> 2 -> 4 -> 1 -> 6 -> 5\\
3 -> (2, 3)\\
2 -> (4, 5)\\
4 -> (6, 2)\\
\section{11.20}
\label{sec:org2b16aa7}
If the graph has two or more edges with the same weight it is possible\\
to produce different trees. Each algorithm has a point at which they\\
must choose some edge with the minimum weight and if two or edges have\\
the same weight this choice is ambiguous. Depending on the\\
implementation of each respective algorithm there may be a different\\
choice of edge. This would result in a different MST. If the graph had\\
all unique weights then the MST would be necessarily unique and the\\
two algorithms would always produce the same MST.\\
\section{11.22}
\label{sec:org8fee175}
Yes both work fine with negative weights. In Prim's, the least weight\\
edged connected to a vertex not in the tree is added. If that edge\\
happens to have a negative weight that is not a problem. The algorithm\\
just continues. Likewise, In Kruskal's, the least weight edge that\\
connects two components is added. If that is negative then that isn't\\
a problem.\\
\section{11.23}
\label{sec:org7b6c33b}
Dijkstra's algorithm produces a tree that contains the shortest path\\
from a source vertex to every other vertex that is reachable in the\\
graph. If the graph is connected then this is a tree that contains\\
every vertex so it is a spanning tree.  It is not necessarily a\\
MST. Dijkstra's algorithm is concerned with minimizing the path\\
between the source vertex and every other vertex. A MST is a tree with\\
the minimum total weight. The minimum weight from a source to another\\
vertex may not result in a minimum weight in the MST. Consider the\\
following graph.\\
1 -> (2, 5) -> (3, 5) -> (4, 5)\\
2 -> (1, 5) -> (4, 1)\\
3 -> (1, 5)\\
4 -> (1, 5) -> (2, 1)\\
Let the source vertex be 1. Dijkstra's algorithm would find the\\
shortest path from 1 to 4 is just the edge connecting 1 and 4 but the\\
MST would not have this edge. The MST would have the edges 1 to 3, 1\\
to 2 and 2 to 4. So Dijkstra's algorithm does not necessarily produce\\
an MST.\\
\section{7.4}
\label{sec:orgd22b727}
It would reduce the number of comparisons made as the binary search on\\
the sorted part would find the position to place the element in at\\
most log\(_{\text{2}}\)(n) with n equal to the length of the sorted part of the\\
array. It would not change the asymptotic running time of the\\
algorithm however because the swaps still have the same running time\\
so it is still O(n\(^{\text{2}}\)).\\
\section{7.6}
\label{sec:org23f1dbe}
\subsection{Insertion Sort}
\label{sec:org7aa3df5}
Insertion Sort is stable because if an element that appears before the element\\
to be inserted is equal to the element to be inserted, the comparison in the\\
algorithm will dictate that it is never be inserted before that element.\\
\subsection{Bubble Sort}
\label{sec:org4340dfe}
Bubble sort is stable because the algorithm progressively compares elements\\
in the array and only swaps them if there is an inversion. So this algorithm\\
will always maintain the relative ordering of elements that are equal.\\
\subsection{Selection Sort}
\label{sec:org177af6d}
Selection sort is not stable because the algorithm swaps the minimum element in\\
the unsorted part of the array with the first position in the array and this may\\
invert the order of two equal elements. The algorithm can be stable if instead\\
of swapping, it inserts the minimum element at the first position in the\\
unsorted array and pushes the following elements back one position. This would\\
guarantee that the relative ordering of equal elements would remain.\\
\subsection{Shell sort}
\label{sec:orgc9c92f5}
Shell sort is not stable because the algorithm relies on swapping non-adjacent\\
elements. This may result in two equal elements inverting their position if they\\
are in separate sub lists. It can't be made stable because the non-adjacent\\
swapping is the fundamental reason why the algorithm works.\\
\subsection{Merge sort}
\label{sec:org1ca6045}
Merge sort is stable because when merging lists, two elements are swapped only\\
if the left value is strictly greater than the right value. This would leave the\\
equal values in their original order.\\
\subsection{Quick sort}
\label{sec:orga5bfc36}
Quick sort is not stable because the partition algorithm swaps non-adjacent\\
elements and most partition schemes would swap a value that is equal to the\\
pivot with an element that is less than the pivot if those values were in\\
positions that could be swapped by the partition algorithm. The algorithm could\\
be made stable by adding some logic in the partition algorithm that checks if\\
the index of the equal element was less than the resulting pivot position. If\\
its index is less then it is placed in the lesser partition otherwise it is\\
place in the greater partition.\\
\subsection{Binsort}
\label{sec:org23d2fa5}
Binsort is stable because everytime an element of equal value to an\\
element that appeared earlier in the list occurs it is appended to the\\
bin. So relative ordering of the equal elements is always preserved.\\
\subsection{Radix sort}
\label{sec:org40976db}
Radix sort is not stable because it places values into the bin from\\
the bottom of the bin up then replaces into the array from the top\\
down. This does not preserve order. I cannot be made stable.\\
\section{7.9}
\label{sec:orgbfdc510}
The worst case for quick sort occurs when a maximum or minimum element\\
is chosen as a pivot. The findPivot method implemented in 7.5 choses\\
the median index as its pivot so any permutation with 0 or 7 at\\
position 3(zero-origin indexing) will cause a worst case to occur.\\

[0 1 2 7 3 4 5 6]\\
\section{7.19}
\label{sec:orga9cfe62}
Notice that for n calls to merge sort, you first call mergesort on one\\
list, then two list, then 4 lists and so on\\

1 + 2 + 4 + 8 +\ldots{} = \(\sum_{\text{i=0}}^{\text{n}}\) 2\(^{\text{i}}\) = 2\(^{\text{n+1}}\) - 1\\

When mergsort is called on the final set of lists of size 2\(^{\text{n}}\) in this\\
series, insertion sort is run instead. So the amount of insertion\\
sorts run is\\

2\(^{\text{n}}\) = (n + 1)/2\\
\end{document}