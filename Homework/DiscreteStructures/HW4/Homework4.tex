% Created 2019-04-21 Sun 14:36
% Intended LaTeX compiler: pdflatex
\documentclass[11pt]{article}
\usepackage[utf8]{inputenc}
\usepackage[T1]{fontenc}
\usepackage{graphicx}
\usepackage{grffile}
\usepackage{longtable}
\usepackage{wrapfig}
\usepackage{rotating}
\usepackage[normalem]{ulem}
\usepackage{amsmath}
\usepackage{textcomp}
\usepackage{amssymb}
\usepackage{capt-of}
\usepackage{hyperref}
\author{Parker Gabel}
\date{\textit{<2019-03-25 Mon>}}
\title{Homework3}
\hypersetup{
 pdfauthor={Parker Gabel},
 pdftitle={Homework3},
 pdfkeywords={},
 pdfsubject={},
 pdfcreator={Emacs 26.1 (Org mode 9.1.9)}, 
 pdflang={English}}
\begin{document}

\maketitle
\section{5.13}
\label{sec:orgff174d5}
\subsection{}
\label{sec:org200df3c}
D = 4 and P = 4. The overhead for the tree is n3P = n12 and the total\\
space for the tree is n(3P + D) = n16. So the overhead fraction is\\
n12/n16 = 12/16= 3/4\\
\subsection{}
\label{sec:org2c45299}
D = 16 and P = 4. The overhead for the tree is n2P = n8 and the total\\
space for the tree is n(2P + D) = n(8 + 16) = n24. So the overhead\\
fraction is n8/n24 = 1/3\\
\subsection{}
\label{sec:orgde8feba}
Let n be the number of internal nodes. By the Full Binary Tree theorem\\
there are n + 1 leaves. So D = 8 and P = 4. The overhead for internal\\
nodes is n3P = n12 and the overhead for the leaves is (n + 1)P =\\
n4 + 4. The total overhead for the tree then is n12 + n4 + 4 = n16 + 4\\
= 4(n4 + 1). The total space for the internal nodes is n(3P + D) = n20\\
and the total space for the leaves is (n + 1)(P + D) = (n + 1)12. So\\
the total space is n20 + n12 + 12 = n32 + 12 = 4(n8 + 3). The overhead\\
fraction then is 4(n4 + 1)/4(n8 + 3) = (n4 + 1)/(n8 + 3).\\
\subsection{}
\label{sec:orgdbd3b43}
Let n be the number of internal nodes. By the Full Binary Tree theorem\\
there are n + 1 leaves. So D = 8 and P = 4. The overhead for internal\\
nodes is n2P = n8 and the overhead for the leaves is 0. The total\\
overhead for the tree then is n8 + 0 = n8. The total space for the\\
internal nodes is n2P = n8 and the total space for the leaves is (n +\\
1)D = 8(n + 1). So the total space is n8 + n8 + 1 = n16 + 1. The\\
overhead fraction then is n8/(n16 + 1).\\
\subsection{}
\label{sec:orgcdbc5ce}
\subsubsection{120 :left 42 :right null}
\label{sec:org87f0d6e}
\subsubsection{42 :left 2 :right 42}
\label{sec:org7c29cab}
\subsubsection{2 :left null :right 32}
\label{sec:org799900b}
\subsubsection{32 :left 24 :right 37}
\label{sec:orgc69de1c}
\subsubsection{24 :left null :right null}
\label{sec:org22d4eb5}
\subsubsection{37 :left null :right 40}
\label{sec:org746dccd}
\subsubsection{40 :left null :right null}
\label{sec:org363f540}
\subsubsection{42 :left null :right null}
\label{sec:org52b916f}
\end{document}
